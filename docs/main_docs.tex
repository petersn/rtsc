\documentclass[10pt]{book}
\usepackage{amsmath,amssymb,hyperref}
\usepackage[margin=3cm]{geometry}
\usepackage{listings}
\lstset{
  language=C,                % choose the language of the code
%  numbers=left,                   % where to put the line-numbers
  stepnumber=1,                   % the step between two line-numbers.        
  numbersep=5pt,                  % how far the line-numbers are from the code
  showspaces=false,               % show spaces adding particular underscores
  showstringspaces=false,         % underline spaces within strings
  showtabs=false,                 % show tabs within strings adding particular underscores
  tabsize=4,                      % sets default tabsize to 2 spaces
  captionpos=b,                   % sets the caption-position to bottom
  breaklines=true,                % sets automatic line breaking
  breakatwhitespace=true,         % sets if automatic breaks should only happen at whitespace
%  title=\lstname,                 % show the filename of files included with \lstinputlisting;
}
\title{\texttt{RTSC} Manual}
\author{Peter Schmidt-Nielsen}
\begin{document}
\frontmatter
\thispagestyle{empty}
\begin{center}
\vspace*{5cm}
{\Huge \texttt{RTSC} Manual} \\
\vspace{2cm}
{\Large Peter Schmidt-Nielsen}
\end{center}
\tableofcontents
\chapter{RTSC}
\section{Overview}
The RTSC compiler works as follows:
\begin{enumerate}
\item
A project file (\texttt{.rtsc-proj}) is read in.
The file format is a standard .ini file, as parsed by Python's ConfigParser module.
\item
The section \texttt{[config]} must be present, and must contain a variable \texttt{main\_file}.
This file is loaded, and compiled, recursing on any import statements.
The resulting output is Javascript.
\item
The standard Javascript header (found at \texttt{rtsc/libs/std.js}) is prepended to the compiled Javascript, completing the Javascript executable.
\item
Files in the \texttt{[files]} section are included, as outlined in \ref{section_files}.
\item
The resultant files and Javascript are quick-linked (see \ref{section_quick_links}) into the output binary.
\end{enumerate}

\section{Quick Links}
\label{section_quick_links}
In order to not need to distribute a compiler or linker of any sort, RTSC has a built-in compilation mechanism called quick links.
For each target platform, a pair of files are stored under \texttt{rtsc/quick\_links}, namely: \texttt{target\_data} and \texttt{target\_relocs}.
%The file \texttt{target_data} contains the base quick links binary to be used.
Quick linking proceeds as follows:
\begin{enumerate}
\item
An rtscfs image (see \ref{section_rtscfs}) is created containing the game data.
The entry \texttt{js} in this image contains the Javascript string (not null-terminated) to be executed.
\item
The file \texttt{target\_data} is read in.
The file \texttt{target\_relocs} contains a sequence of relocation instructions of the form \texttt{"address,length,var\textbackslash n"}.
Each relocation points to into \texttt{target\_data}, and specifies a variable to be added into the bytes from \texttt{address} to \texttt{address}+\texttt{length}, in little endian format.
Currently, only the variable \texttt{fs\_size} is supported, and \texttt{fs\_size} equals the length of the rtscfs image.
An example actual relocation file for a particular revision of the elf64 target:
\lstinputlisting{elf64_relocs}
\item
The output binary is the relocated contents of \texttt{target\_data} concatenated with the rtscfs image for the binary.
The quick links binaries have been made to expect data to be added at their ends.
\end{enumerate}

\section{File Inclusion}
\label{section_files}
Variable in the \texttt{[files]} section of the project file being compiled cause files to be included into the end binary.
The syntax is:
\begin{center}
\texttt{var = [specifier::]path}
\end{center}
The data can be loaded from within RTCS with code \texttt{\_load\_fs\_data("var")}.
If a specifier is given, it causes preprocessing of the data in \texttt{path}.
Valid specifiers are:
\begin{center}
\begin{tabular}{c | l | l}
	Name & Function & Loads with \texttt{\_load\_fs\_data} as \\\hline
	\texttt{texture} & Loads texture data from an image file. & Special buffer object. \\
	\texttt{config} & Parses the file as a .ini. & \texttt{\{"section":\{"var":"value",...\},...\}} \\
	\texttt{json} & Parses the file as JSON. & Corresponding Javascript object. \\
	\texttt{string} & Stores the path as the datum. & String object. \\
\end{tabular}
\end{center}
If no specifier is given, the file is loaded unprocessed, and \texttt{\_load\_fs\_data} yields a string object containing the file's contents.
All data included into the binary is compressed with bzip2 if doing so would render it smaller.

\section{rtscfs}
\label{section_rtscfs}
RTSC binaries use a built-in filesystem called rtscfs to store data along with the binary.
The rtscfs header is:
\lstinputlisting{rtscfs_header}
The field \texttt{fs\_offset} points to contiguous array \texttt{fs\_count} long of the following structure:
\lstinputlisting{rtscfs_entry}
All offsets are relative to the beginning of the original \texttt{rtscfs\_header\_t}.
Each entry points to its own name with the fields \texttt{entry\_name\_size} and \texttt{entry\_name\_offset}.
The entry name pointed to is guaranteed not to contain any null bytes, but is \emph{not} guaranteed to be null-terminated as placed in the file!
The field \texttt{entry\_name\_offset} has no alignment restrictions.
To achieve a directory structure, an entry name should contain slashes.
It is recommended that creators of rtscfs images do \emph{not} prefix every entry name with a slash.

The data contained by an entry is \texttt{entry\_size} long and pointed to by \texttt{entry\_offset}.
The field \texttt{entry\_offset} must be aligned to a 16-byte boundary.

The following flags may be listed:
\begin{center}
\begin{tabular}{ c | c }
	Name & Value \\\hline
	\texttt{RTSCFS\_FLAG\_BZ2} & 0x1
\end{tabular}
\end{center}
If the flag \texttt{RTSCFS\_FLAG\_BZ2} is set, then the entry's data must be a \texttt{uint64\_t} decompressed length field followed by a valid bzip2 compressed file, complete with magic number and all.
The decompressed length field sets the size of the buffer allocated to decompress the file into -- it must be set correctly!
The data are decompressed transparently, and thus the \texttt{RTSCFS\_FLAG\_BZ2} flag may be set or not on any entry of any type without error.

\end{document}
